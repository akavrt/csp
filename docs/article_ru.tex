\documentclass[12pt]{article}
\usepackage[utf8x]{inputenc}
\usepackage[T2A]{fontenc}
\usepackage[english,russian]{babel}
\usepackage[a4paper]{geometry}
\geometry{hmargin={3cm, 2cm}}
\geometry{vmargin={2cm, 2cm}}
\usepackage{indentfirst}
\usepackage{amsmath}
\sloppy

\title{Эволюционный алгоритм решения задачи \\
рационального раскроя рулонного материала}

\author{В.~Н.~Балабанов\thanks{E-mail: \texttt{akavrt@gmail.com}}, 
Ю.~А.~Скобцов \\ [1.0ex]
\normalsize\textit{Донецкий национальный технический университет}
}

\date{Препринт: \today}

\begin{document}

\maketitle

\begin{abstract}
Рассматривается задача рационального раскроя рулонного материала в 
многокритериальной постановке. Уточнения, внесенные в формальную модель, 
отражают необходимость учитывать технологический аспект при составлении планов 
продольного раскроя рулонов. Предложенный метод решения задачи рационального 
раскроя основан на оптимизационном аппарате эволюционных вычислений. При выборе 
способа представления решений в эволюционном алгоритме учтена комбинаторная 
природа исходной задачи. Комбинаторный подход при разработке эволюционных 
операторов усилен применением адаптивной генерации раскройных карт. 
Подготовлены тестовые задачи и поставлен ряд вычислительных экспериментов. 
Полученные результаты подтверждают состоятельность избранного подхода к 
построению эволюционного алгоритма, который может быть адаптирован для решения 
других задач рационального раскроя в схожей постановке.
\smallskip

\noindent\textit{Ключевые слова:} рациональный раскрой, 
рулонные материалы, многокритериальная модель, комбинаторная оптимизация, 
эволюционный алгоритм.
\end{abstract}    

\section{Введение}

Задача о рациональном раскрое впервые была сформулирована выдающимся советским 
математиком и экономистом Л.~В.~Канторовичем [1] 
и состоит в нахождении такого плана раскроя, который позволяет получить 
заготовки в нужном ассортименте и при этом обеспечивает минимальный расход 
материала. В общем случае задача о рациональном раскрое может быть 
классифицирована как NP-сложная задача комбинаторной оптимизации. 
Логическая структура раскройных задач во многом определяется формой и размерами 
применяемых заготовок, а также ассортиментом исходного материала [2]. 
В частности, различают одно- и двухмерный, единичный и комбинированный, 
прямоугольные произвольный и гильотинный, а также фигурный раскрои.

План, являющийся решением задачи рационального раскроя, представляет собой 
перечень раскройных карт, которые определяют способы раскроя отдельных частей 
исходного материала (например, рулонов или листов) на наборы заготовок. При 
построении формальных моделей задач рационального раскроя наибольшее 
распространение получил поход, основанный на сведении исходной комбинаторной 
задачи к задаче целочисленного линейного программирования общего вида [3]. 
Критерий оптимальности формулируется таким образом, чтобы при поиске решений 
минимизировать расход, стоимость или потери исходного материала, 
использованного в раскрое. Актуальным направлением является построение 
уточненных многокритериальных моделей, в которых более полно учитываются 
ограничения, налагаемые конструкцией раскройного оборудования, либо раскрой 
рассматривается как одна из стадий технологического процесса производства 
некоторой продукции. Наряду с расходом материала наиболее часто требуется 
минимизировать общее количество настроечных операций, производимых с 
оборудованием в ходе выполнения плана раскроя [4, 5]. Многокритериальные 
модели, в которых совместно рассматривается более двух различных критериев, 
встречаются значительно реже [6].

В решении задач рационального раскроя широко используются как точные, так и 
приближенные методы [7, 8]. Применимость точных методов до последнего времени 
была ограничена задачами в классической постановке, предполагающей рассмотрение 
только одного оптимизационного критерия~--- минимизации расхода исходного 
материала. Для решения практических задач рационального раскроя был предложен 
ряд эвристических методов, которые позволяют находить решения, удовлетворяющие 
требованиям реального производства [4, 9–-13].

Рулонные материалы широко применяются в металлургической, целлюлозно-бумажной, 
текстильной, химической и целом ряде других отраслей промышленности. В данной 
статье рассматривается задача рационального раскроя рулонного материала, 
выполняемого с помощью специализированного оборудования~--— линий продольной 
резки рулонов. При планировании раскроев такого типа требуется учитывать 
дополнительные критерии оценки качества получаемых решений и ограничения, 
связанные с особенностями технологического процесса продольного раскроя 
рулонов [4, 11]. Изложение материала статьи организовано следующим образом: 
вначале дается уточненная формальная модель задачи рационального раскроя 
рулонов на линиях продольной резки, далее подробно рассматриваются различные 
аспекты построения эволюционного алгоритма, обсуждаются результаты 
вычислительных экспериментов и делаются соответствующие выводы.


\section{Постановка задачи рационального раскроя}

При раскрое рулонов на линиях продольной резки материал раскраивается на полосы 
заданной ширины, все продольные резы выполняются от края до края, параллельно 
боковой кромке исходной полосы. Следовательно, продольный раскрой рулонного 
материала можно классифицировать как одномерный. В тоже время, в отличие от 
классической постановки одномерной задачи рационального раскроя, основанной на 
поштучном способе учета комплектности раскроя (когда требуемое количество 
заготовок каждого типа задается целым числом), при продольном раскрое рулонов 
может указываться только требуемая общая длина или масса полос заданной ширины, 
но не их точное количество. В некоторых источниках такой вид раскроя получил 
название полуторамерного [11].

Введем ряд обозначений. Положим, что ассортимент заготовок, которые требуется 
получить в результате раскроя рулонов, ограничен $m$ наименованиями. Размеры 
заготовок и комплектность раскроя задаются следующим образом: для получения 
заготовки длиной $l_i$ допускается использование произвольного количества 
отдельных полос шириной $w_i$, $i=1,\ldots,m$. Пусть исходный материал 
представлен перечнем из $n$ рулонов, с каждым из которых сопоставлен порядковый 
номер $j=1,\ldots,n$. Пригодные для использования длину и ширину рулона с 
номером $j$ обозначим через $L_j$ и $W_j$, соответственно.

Множество раскройных карт опишем с помощью целых чисел $a_{ik}$, которые 
определяют количество полос шириной $w_i$, полученное при раскрое рулона по 
способу с номером $k$. Для того чтобы установить связь между раскройными 
картами и рулонами, введем булевы переменные $T_{jk}$:
\[ T_{jk}=\begin{cases}
             1, & \text{рулон } j \text{ раскраивается по способу } k, \\
             0, & \text{в противном случае}.
          \end{cases} \]
Если предположить, что все допустимые раскройные карты перечислены по 
$k=1,\ldots,K$, то задача нахождения рационального плана, который позволит 
раскроить рулоны на наборы полос заданного размера с минимальными потерями 
исходного материала, может быть сформулирована следующим образом:
\begin{equation}
    z_1=\min{\sum_{k=1}^{K} \sum_{j=1}^{n} T_{jk} L_j 
        \left(W_j-\sum_{i=1}^{m} a_{ik}w_i\right)}
\end{equation}
\begin{equation}
     \sum_{k=1}^{K} \sum_{j=1}^{n} T_{jk} a_{ik} L_j \geq l_i \quad \forall i
\end{equation}
\begin{equation}
     T_{jk} \sum_{i=1}^{m} a_{ik} w_i \leq W_j \quad \forall i,k
\end{equation}
\begin{equation}
     \sum_{k=1}^{K} T_{jk} \leq 1 \quad \forall j
\end{equation}
\[ i \in \{1,\ldots,m\}, \: j \in \{1,\ldots,n\}, \: k \in \{1,\ldots,K\}.\]
Известно, что операции настройки отдельных агрегатов линии продольной резки 
рулонов, выполняемые при смене раскройной карты, характеризуются значительной 
трудоемкостью [4, 11]. Повышение качества составляемых планов раскроя может 
быть достигнуто за счет повторного использования раскройных карт: при раскрое 
двух и более рулонов на одинаковые наборы полос количество необходимых 
настроечных операций сокращается, поскольку смена раскройной карты в таком 
случае не требуется. Для того чтобы формализовать данный критерий, введем 
вспомогательную функцию (5).
\begin{equation}
    \delta\left(k, \sum_{j=1}^{n} T_{jk}\right)=
        \begin{cases}
            1, & \text{если } \sum_{j=1}^{n} T_{jk}>0, \\
            0, & \text{в противном случае}.
        \end{cases}    
\end{equation}
Критерий оптимальности для задачи минимизации количества различных раскройных 
карт, используемых в плане раскроя, может быть записан следующим образом:
\begin{equation}
    z_2=\min{\sum_{k=1}^{K}} \delta\left(k, \sum_{j=1}^{n} T_{jk}\right).
\end{equation}
Следовательно, задача рационального планирования продольного раскроя рулонного 
материала может рассматриваться как многокритериальная задача комбинаторной 
оптимизации с векторным критерием (7) и системой линейных ограничений (2)--(4).
\begin{equation}
    Z=(z_1,z_2)
\end{equation}
Зачастую при рассмотрении задач рационального раскроя в подобной либо схожей 
постановке вместо многокритериальной решается задача скалярной оптимизации: 
вводятся коэффициенты, характеризующие относительную важность частных 
критериев, а векторный критерий заменяется соответствующим ему обобщенным 
критерием [4, 11, 14]. Эволюционные метаэвристические методы успешно 
применяются для решения задач комбинаторной оптимизации, в том числе и задач 
рационального раскроя [12]. С учетом комбинаторной сложности рассматриваемой 
задачи выбор эволюционного алгоритма в качестве общей схемы для построения 
приближенного метода решения задачи рационального раскроя рулонов 
представляется логичным.


\section{Эволюционый алгоритм}

Метаэвристические методы, которые при поиске решений параллельно работают с 
несколькими решениями, условно выделяют в самостоятельный класс популяционных 
метаэвристик. Наибольшее распространение из числа методов, относящихся к 
данному классу, получили эволюционные алгоритмы, которые для эффективной 
организации поиска решений оптимизационных задач заимствуют ряд базовых 
принципов дарвиновской теории эволюции [15]. В зависимости от особенностей 
решаемой задачи может использоваться дополнительное кодирование решений, когда 
с помощью некоторого правила решения преобразуются в "<особи">~--- структуры 
данных, специфичные для той или иной разновидности эволюционного алгоритма. 
На этапе инициализации алгоритма формируется начальная популяция заданного 
размера, составленная из особей, сгенерированных случайным образом или же 
полученных с помощью вспомогательной процедуры. Каждая итерация эволюционного 
алгоритма состоит из последовательного выполнения эволюционных операторов, 
изменяющих структуру особей в текущей популяции, и формирования новой популяции 
при помощи механизма, имитирующего естественный отбор: в новую популяцию 
переносятся лишь наиболее приспособленные особи из текущей популяции. Степень 
приспособленности определяется значением целевой функции, рассчитанным для 
решения, соответствующего оцениваемой особи. Например, при решении задач 
минимизации вероятность попадания в следующую популяцию выше у особей с 
меньшими значениями целевой функции. Таким образом, для построения 
эволюционного алгоритма решения оптимизационной задачи требуется выбрать 
подходящий способ представления решений в виде особей, сформулировать целевую 
функцию, разработать эволюционные операторы и выполнить настройку параметров 
алгоритма.

При поиске решения рассматриваемой задачи требуется из множества исходных 
рулонов $S$ выделить подмножество $S_{\text{used}}$, содержащее только те 
рулоны, которые будут раскроены, а также определить способы раскроя этих 
рулонов. В обозначениях формулировки (1)--(7):
\[ S=\{1,\ldots,m\},\: 
   S_{\text{used}}=\left\{ j \in S \: \bigm| \: \sum\nolimits_{k=1}^{K} T_{jk}=1 \right\}. \]
Следует отметить, что множество допустимых раскройных карт в формулировке 
(1)--(7) задано неявно, поскольку на практике задача генерации и перебора всех 
его элементов является трудноосуществимой вследствие значительной мощности 
множества. Исходя из этого, предлагается применить адаптивную генерацию 
раскройных карт, выполняемую по мере необходимости в процессе поиска решений 
с помощью специальной процедуры. На эволюционный алгоритм при этом возлагаются 
функции по подбору рулонов, вызову процедуры генерации раскройных карт и 
управлению общей структурой решения.

\subsection{Способ представления решений}

Такие разновидности эволюционных алгоритмов, как генетические алгоритмы и 
эволюционные стратегии, изначально разрабатывались как методы решения задач 
численной оптимизации~--- способы представления решений таких задач, 
предложенные в ранних работах, со временем стали считаться стандартными. 
Совершенно иначе обстоит дело при решении задач комбинаторной оптимизации, 
когда выбор правила, в соответствие с которым решения задачи преобразуются в 
особи, определяется логической структурой решаемой задачи.

Как уже было отмечено, при составлении плана продольного раскроя рулонов 
требуется учитывать не только потери материала, но также контролировать 
количество введенных в план различных способов раскроя рулонов. Удобным для 
реализации эффективных эволюционных операторов представляется такой способ 
представления решений, при котором отдельные рулоны, включенные в план, 
будут объединяться в группы $G_q$, каждой группе рулонов будет соответствовать 
собственный, отличный от других групп способ раскроя материала. Предположим, 
что номера раскройных карт, используемых в плане, известны и равняются $P_q$, 
$q=1,\ldots,Q,$ тогда задача разбиения множества рулонов $S_{\text{used}}$ 
на группы $G_q$ может быть сформулирована следующим образом:
\[ G_q=\{ j \in S_{\text{used}} \: \bigm| \: T_{jk}=1,k=P_q \}, \: 
   P_q \in \{1,\ldots,K\}, \: 
   \sum\nolimits_{k=1}^{K} \delta\left(k, \sum\nolimits_{j=1}^{n} T_{jk}\right)=Q, \]
\[ \bigcup\nolimits_{q=1}^{Q}G_q=S_{\text{used}}, \: 
   G_q \cap G_r=\emptyset \text{ при } q \neq r \text{ и } q,r \in \{1,\ldots,Q\}.
\]
Если изменения, вносимые в особь эволюционным оператором, изолируются в 
пределах некоторой группы $G_q \rightarrow G_q^*$ и $|G_q^*| \geq |G_q|$, то 
значение $Q$ после выполнения такого оператора остается неизменным. Применение 
в эволюционном алгоритме операторов, которые соответствуют сформулированному 
выше требованию, упростит поиск решений с нужными характеристиками за счет 
более эффективного контроля значений критерия (6).

\subsection{Целевая функция}

Решение задачи рационального раскроя в многокритериальной постановке (1)--(7) 
предполагает использование математического аппарата Парето-оптимальности. 
Однако, при решении большинства практических задач рационального раскроя 
имеется возможность сопоставить стоимость единицы площади рулонного материала 
со средней трудоемкостью выполнения операций по настройке агрегатов линии 
продольной резки рулонов на заданную раскройную карту. Следовательно, система 
предпочтений лица, принимающего решение, в данном случае~--- составителя планов 
раскроя, может быть описана с помощью коэффициентов относительной важности 
$C_1$ и $C_2$ частных критериев (1) и (6), соответственно. Заменим векторный 
критерий (7) обобщенным критерием (8), который будем использовать в качестве 
целевой функции в эволюционном алгоритме.
\begin{equation}
    Z^*=\min{(C_1z_1^*+C_2z_2^*)},
\end{equation}
\[C_1 \leq 0, \: C_2 \leq 0, \: C_1+C_2=1.\]
\end{document}